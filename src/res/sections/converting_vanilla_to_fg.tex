The \textit{fusion-girl-vanilla-refits} channel on the \href{https://discord.gg/9vBqB9}{ZeX Discord} is the working channel
for the \textbf{official} Fusion Girl Vanilla Outfit Refits. This section is for all who want to help with the 
project. Be sure to check the pinned messages of the channel before continuing so you know the setup, naming schema 
and rules for this project.\\
\textbf{Naming rules:}\\
\textit{Display Name:} FusionGirl-Vanilla-NameOfOutfit-ArmorPiece\\
\textit{Slider Set File:} FusionGirl-Vanilla-NameOfOutfit\\
\textit{Shape Data Folder:} FusionGirl-Vanilla\\
\textit{Shape Data File:} FusionGirl-Vanilla-NameOfOutfit-ArmorPiece\\
The name of the outfit equals the name of the folder you import the mesh from and if you have any additional armor pieces like 
gloves or \textit{F\char`_Arm\char`_Heavy\char`_L} than append a \textit{-ArmorPiece} to the Display Name and Shape Data File. Example:\\
\textit{Display Name:} FusionGirl-Vanilla-CombatArmor-ArmHeavyL\\
\textit{Slider Set File:} FusionGirl-Vanilla-CombatArmor\\
\textit{Shape Data Folder:} FusionGirl-Vanilla\\
\textit{Shape Data File:} FusionGirl-Vanilla-CombatArmor-ArmHeavyL\\
Another example:\\
\textit{Display Name:} FusionGirl-Vanilla-ArmoredCoat\\
\textit{Slider Set File:} FusionGirl-Vanilla-ArmoredCoat\\
\textit{Shape Data Folder:} FusionGirl-Vanilla\\
\textit{Shape Data File:} FusionGirl-Vanilla-ArmoredCoat\\
The vanilla meshes can be found inside \textit{Fallout4 - Meshes.ba2} and if you do the DLC in the \textit{DLC - Main.ba2} file.
You will need a tool to extract those meshes, I will use \href{https://www.nexusmods.com/fallout4/mods/78}{B.A.E}. I created a 
Vanilla to Fusion Girl conversion reference that is available in the misc section of the mod page.\\
When you have everything ready open \textit{bae.exe} and \textit{File->Open File}. Click \textit{Select None} to deselect everything
because we only want some meshes. Inside \textit{Meshes} you should watch out for \textit{Armor} and \textit{Clothes}. 
\href{https://tinyurl.com/y2podp84}{Here} is a spreadsheet containing information about what outfits are already done and what 
needs a conversion. If you decide to work on an outfit please comment in the notes field of that outfit that you are working on it.
You can select those and hit extract if you want all outfit meshes or expand one of them and select the outfit you want to convert. 
When hitting \textit{Extract} select a folder and let the program do its magic.\\
Start OS and select \textit{File->Import->From NIF}. You will have a lot of meshes to select from so let me explain what they do:
\begin{itemize}
    \item \textbf{GO\char`_\dots} - don't bother
    \item \textbf{\dots\char`_1stPerson} - don't bother
    \item you want everything else
\end{itemize}
The rest of this section depends on what type of outfit you loaded.\\
\textbf{Clothes:}\\
When selecting what mesh to load, you will notice that you can choose between 
\textit{OutfitF.nif}, \textit{OutfitM.nif}, \textit{FOutfit.nif} and \textit{MOutfit.nif}. The \textit{F} and \textit{M} stand for 
the sex so load the mesh with the \textit{F} in the name.\\
These outfits are mostly complete outfits for slot 33. If you haven't read the Biped Slots section go there and read what slot 33 means.
After loading the outfit in OS you will notice a mesh called \textit{BaseFemaleBodyFitted:0} which you have to delete. I recommend 
splitting the mesh because currently you only have one mesh with all parts (eg Shoes, Shirt, Pants,...). I always seperate at least 
the shoes so select the mask tool and mask those shoes. Once masked select the mesh and \textit{Rigth Click->Seperate Vertices} call it
\textit{Shoes}.\\
Vanilla outfits are low-poly which can be a pain to work with meaning that you have the option to increase the amount of vertices.
You will need Blender for this step so go ahead and start Blender. In Outfit Studio select the mesh you want to edit, which should be
everything but the shoes, and \textit{Right Click->Export->To OBJ}. Go into Blender and import the \textit{.obj} file using 
\textit{File->Import->Wavefront (.obj)}. Select the mesh in the top right corner and hit \textit{Tab} to go into edit mode.
Select \textit{Edge->Subdivide} in the menu that is at the top of the viewport (see the picture in section 4.1 if you're stuck). 
Make sure you still have the mesh selected and press
\textit{Tab} to go into Object Mode again. Exporting the mesh is as easy as going into \textit{File->Export->Wavefront (.obj)}
but use the correct settings! Those can be changed in the botton left corner. You will have to tick \textit{Selection Only} and untick 
\textit{Write Materials}! After Exporting go into Outfit Studio again and \textit{File->Import->From OBJ}. The name of the imported mesh 
has to be unique so just append a \textit{-subdivided} or something to the name. The mesh doesn't have a material applied to it so 
copy the material from the original mesh by selecting it and \textit{Right Click->Properties} and copy pasting the Material path.
The bone weights also have to be copied over so select the original mesh and \textit{Right Click->Set Reference} than select the subdivided 
mesh and \textit{Right Click->Copy Bone Weights}. Try pressing \textit{Ctrl+Shift+S} to see if there are unweighted areas, if so than 
copy the bone weights again but with a bigger search radius (increment in 0.5 or 1 steps).\\
When you have copied the Material and the Bone Weights from the original mesh to the subdivided mesh you can delete the original as 
we will not need it anymore. The next steps are similar to your normal conversion so go into the \textit{Bones} tab and delete all 
non cloth bones from the project. Instead of loading the \textit{CBBE To Fusion Girl} reference model we of course have to load the 
\textit{Vanilla To Fusion Girl} reference. Select the main mesh (not the shoes) and hit \textit{Ctrl+C} to conform it and set the slider 
to 100 and select \textit{Shape->Set Base Shape}. Load the Fusion Girl reference model and begin your normal conversion process.\\
Once you are done with doing the normal conversion steps and you want to save, make sure that you follow the naming rules available in 
the \textit{fusion-girl-vanilla-refits} channel on the \href{https://discord.gg/9vBqB9}{ZeX Discord}. After you are done with converting, 
testing and adjusting go to the \href{https://tinyurl.com/y2podp84}{spreadsheet} and comment on the outfit you just did that you are done.
You will have to upload the BodySlide files somewhere and provide a link for us to download. Do not upload to the Nexus but to a provider 
such as GDrive, Dropbox, Mega,...\\
\textbf{Armor:}\\
There are two types of Armor you will find: Individual armor pieces and slot 33 armor. Example for the first are Gloves, \textit{F\char`_Arm\char`_L}, \textit{F\char`_Arm\char`_R}, \textit{F\char`_Helmet},
\textit{F\char`_Leg\char`_L}, \textit{F\char`_Leg\char`_R}, \textit{F\char`_Torso} while examples for slot 33 armor ar the armored coat and army fatigues. Converting slot 33 armor is the same process as converting clothes 
so follow the \textit{Clothes} part above for those.\\
Converting individual armor pieces is more tedious than some slot 33 outfit because you are not doing one outfit for a long period of time 
but multiple small once. The good thing is that you can skip a lot of armor pieces when done correct.\\
Before loading your armor piece take a look at the naming schema for the other once. You will see that you can have two seperate meshes for 
left and right so take a look and see if they are the same.\\
After loading the mesh go into the \textit{Bones} tab and delete all non cloth bones from the project. Load the \textit{Vanilla To Fusion Girl} 
reference model, conform all, set the slider to 100, set base shape and load the Fusion Girl body. Next up are just normal conversion 
steps but you can skip a lot of sliders (or all depending on the piece) because when doing an arm piece than you won't have to look at the legs.\\
Before you go and save the project make sure you know the naming rules available in the \textit{fusion-girl-vanilla-refits} channel 
on the \href{https://discord.gg/9vBqB9}{ZeX Discord}.