I honestly always avoid creating BodySlide files from a mesh and it's rather rare to not find at least CBBE BodySlide files for an outfit 
released in the last 1-2 years. If you, for some unknown reason in this world, have to convert the mesh of an outfit to Fusion Girl BodySlide files,
than come prepared and experienced. Don't try this if you've only done like five conversions so far. My first apprentice actually 
tried to do this as their first conversion which made me create this section as a warning.\\
If you want to convert an outfit to the Fusion Girl body that was made for CBBE but doesn't come with BodySlide files than 
you have to use the provided mesh directly. This process is often very messy and the result can be very bad if you don't have
enough experience in this.\\
\textbf{Method 1:}\\
Load the outfit into OS using \textit{File->Load Outfit} and give the project a good Display Name like "\textit{NameOfOutfit - Fusion Girl}".
You have to browse for the nif file and can keep the textures settings to \textit{Automatically search for textures}.\\
Once the outfit go to the bones tab and delete all old bones. In the mesh tab you will notice no reference model but depending 
on the type of outfit you may find a body mesh that isn't a reference model. This mesh should be called \textit{CBBE} or \textit{CBBE Body} 
or something similar. If you have this kind of mesh than select it and \textit{Right Click->Set Reference}. The name of the mesh 
should now be {\color{green}{green}}. Next up is your typical process of loading the \textit{CBBEToFusionGirl} reference model, conforming 
all meshes, setting the slider to 100, set base shape and loading the Fusion Girl reference model.\\
Now is the part that gets complicated:\\
The mesh you loaded may have not been made for the zero slidered CBBE preset. You will most likely have not only clipping but also 
gaps between the outfit and the body. You will have to increase the mesh on the parts that clip, like you would normally do, but 
also decrease the mesh to make it fit the zero slider base shape of the Fusion Girl body.\\
You have to work extremely smooth and clean or the mesh will take on a weird shape. This process requires a lot of precision and the use 
of all tools OS has to offer.\\
\textbf{Method 2:}\\
Another method would be to make the outfit fit the CBBE zero slider preset and than convert to Fusion Girl. So you would convert the 
CBBE Mesh to CBBE BodySlide and than to Fusion Girl BodySlide. I don't like this method as it requires you to switch between FG and CBBE 
and take an extra step but it will result in a somewhat cleaner conversion.\\
\textbf{Method 3 (best choice):}\\
This one I picked up doing Vanilla outfits. You will need Blender and the Blenderfiles from the misc section of the mod page. We will make use of the 
\href{https://docs.blender.org/manual/en/latest/modeling/modifiers/deform/shrinkwrap.html}{Shrinkwrap Modifier}. Do note that this 
process is not perfect and the result may vary from outfit to outfit. Go into Outfit Studio and load the outfit you want to convert.
You want to export the outfit and use it in Blender \textbf{but} you have to be careful with what you export. Do not export shoes, gloves, 
stuff with custom bones like skirts and accessories/equipment. You only want pants/short/whatever covers your legs and shirt/bra/bikinitop/whatever covers the torso.
Export the mesh by selecting it and \textit{Right Click->Export->To OBJ}.\\
In Blender import the \textit{.obj} file \textit{File->Import->Wavefront (.obj)} and import the \textit{FusionGirl.obj} file from 
the Blenderfiles you downloaded. Select the outfit mesh in the top right corner and select the wrench icon in the botton right half.
Click \textit{Add Modifier} and select \textit{Shrinkwrap}. Select the \textit{FusionGirl} mesh as the target, set the mode to \textit{Above Surface} 
and set the offset to \textit{0.01} or \textit{0.02} depening on the outfit and personal preference. Once all settings are correct click \textit{Apply}.
You can now decide if you want to smooth the mesh inside Blender or Outfti Studio. Either way you want to export the mesh again once you're done 
so select the mesh in the top right corner and \textit{File->Export->Wavefront (.obj)}.\\
Inside Outfit Studio again go to \textit{File->Import->From OBJ} and give it a unique name. Go to the properties menu of the original mesh 
and copy the Material path over to the new mesh. Set the original mesh as reference and copy the bone weights on the new mesh. Delete the 
original mesh and continue with the normal converting process starting with deleting all non cloth bones from the project.