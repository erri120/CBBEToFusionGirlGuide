\includegraphics[width=\textwidth]{os}
\textit{(You might wanna zoom in a bit to see all colors)}\\
This is Outfit Studio when you start it from BodySlide (button in the bottom right corner of BodySlide).\\
This will be our main tool and the area where we will spent the most time in. It is important to get familiar  
with the tool so you don't waste time trying to figure out where that one menu was.
I will go through each of the areas and talk about the most important functions. Most of them are found in 
other editing software or explain themselves. You also have tooltips for everything that are displayed at the button.\\
\textbf{Menu bar ({\color{blue}{blue}}):} The most important menus are \textit{Shape} and \textit{Slider}. You should make use 
of all shortcuts in the menus! Shortcuts will reduce the time it takes to do conversions by \textbf{a lot}.\\
\textbf{Tool bar ({\color{green}{green}}):} This little bar has all the tools you will need. Press 0 to 5 on the keyboard 
to see what tool will get selected. I always get confused that the first tool is on key 0 :p seems like someone took array 
sizes way too far\dots\\
Some tools are usable only when you edit a mesh, others when editing bone weights. This bar also has buttons for changing the views 
but you should use shortcuts from the \textit{Edit} menu for changing views.
\textbf{Brush Settings ({\color{black}{black}}):} 
This menu is accessible by pressing space bar and allows you to change the settings of the current selected brush (key 1 to 5). 
Each brush has their own settings and only the \textit{Size} property stays the same for every brush.\\
\textbf{Main Windows ({\color{red}{red}}):} \textbf{A} is the renderer view where you see the loaded meshes and will work on them,
\textbf{B} a container for different properties of each tab and \textbf{C} will show all sliders of a reference model.\\
\textbf{FOV slider ({\color{lila}{purple}}):} changes FOV\dots very useful\\
\textbf{Tab list ({\color{yellow}{yellow}}):} We only care about the \textit{Meshes} and \textit{Bones} tabs