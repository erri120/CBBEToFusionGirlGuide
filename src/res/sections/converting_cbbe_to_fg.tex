Now is finally the time to do some converting!\\
You will need the \href{https://www.nexusmods.com/fallout4/mods/37380}{CBBE to Fusion Girl BodySlide Conversion Reference} if 
you want to convert from CBBE to Fusion Girl.\\
Your first conversion should be something easy and simple like \href{https://www.nexusmods.com/fallout4/mods/6935}{Simply Clothes}.
In this step we will convert Simply Clothes from CBBE to Fusion Girl so be sure to download the mod and start Outfit Studio.\\
The mod already comes with BodySlide files for CBBE so inside OS we only need to load these.\\
This is a good time to talk about the structure of OS projects. Inside \textit{data/Tools/BodySlide} are all BodySlide related files.
The three important folders for Conversions are \textit{ShapeData}, \textit{SliderGroups} and \textit{SliderSets}.\\
A SliderSet is a project you can load in OS and comes as a \textit{.osp} or \textit{.xml} file. ShapeData is saved as a \textit{.nif} 
and a \textit{.osd} file and contains the outfit zero-slidered and all information for slider changes. SliderGroups are \textit{.xml} files 
and assign a project to a group like Fusion Girl. A SliderSet is named a Set because you can have multiple outfits in one Set to keep 
everything clean.\\
To load the existing CBBE BodySlide files into OS we load the Project using the Menu (\textit{Files->Load Project}) or the Shortcut (\textit{Ctrl+O}).
The file is called \textit{SimplyClothesCBBE.osp}. With the outfit loaded, OS should look like this:\\
\includegraphics[width=\textwidth]{os_03.png}\\
\begin{wrapfigure}[5]{r}{0.4\textwidth}
    \vspace{-20pt}
    \includegraphics[width=0.4\textwidth]{os_05.png}
\end{wrapfigure}
\textbf{First} we have to remove all old bones. Simply click on the \textit{Bones} tab from the tab list, select the first bone, 
scroll down and use \textit{Shift+Click} on the last bone and \textit{Right Click->Delete->From Project}.\\
\textbf{Second} load the CBBE to Fusion Girl reference using \textit{File->Load Reference}.\\
\includegraphics[width=0.45\textwidth]{os_04.png}\\
\textbf{Third} conform all shapes (\textit{Ctrl+Shift+C} or \textit{Slider->Conform All}), 
set the CBBEtoFusionGirl slider to 100 and Set Base Shape (\textit{Slider->Set Base Shape}).
You should have noticed that changing the slider from 0 to 100 that the outfit starts to change. This is because we conformed all 
shapes prior to setting the slider. If you were to not conform all shapes and change the slider you will see that only the body changes.\\
\textbf{Fourth} load the Fusion Girl reference. Same way as the CBBE to Fusion Girl reference (\textit{File->Load Reference}).\\
You have now converted a CBBE outfit to a Fusion Girl outfit, good job :). \textbf{BUT} we are long from being finished. Only because 
the reference body for the outfit is now the Fusion Girl body doesn't mean the conversion is '\textit{good}'. 
\href{https://www.youtube.com/watch?v=KAHLwAxS7FI}{This is where the fun begins}. Let me introduce you to your arch enemy: clipping.
When parts of the body clip through the outfit mesh and appear in places you don't want them to.\\
A good conversion has the least amount of clipping possible. It is not possible to achieve true 0\% clipping but you can get very close 
so it's not noticeable. We have to use the complete arsenal of OS and possible other tools to reduce clipping.\\
\textbf{Fifth} reduce clipping for the zero slider base shape. Zero-Slidered is a body preset where all sliders you see in the slider panel 
are set to 0. This is a base shape and the foundation for everything else. This also means that we have to do a good job on the base shape.\\
You can navigate the renderer in different ways: \textbf{Hold Right Click} to rotate, use the \textbf{Scroll wheel} to zoom in and out and
\textbf{Hold Scroll wheel} or \textbf{Shift+Hold Right Click} to move the scene.\\
The outfit we convert has some clipping in the boots mesh. To fix that we switch to the \textit{Meshes} tab and select the \textit{boots} 
Mesh. Using the tool with the shortcut 1 we can increase mesh volume on the clipping parts of the boots:\\
\includegraphics[width=0.5\textwidth]{os_06_texture.png}
\includegraphics[width=0.5\textwidth]{os_06_notexture.png}\\
The picture on the right uses the no texture view (\textit{Edit->Enable Textures} or \textit{T}) which is very helpful for finding 
clipping and working with the mesh in general as you don't really need the texture.\\
After making sure that you don't see any clipping, we need to start looking at the mesh of each shape individually. In the mesh 
tab we can make certain meshes visible, invisible and activate their wire frame mode by pressing the icon left of the name.
I prefer to start with the boots so make everything except the {\color{green}{reference model}} invisible and activate wire frame mode for the selected mesh.\\
\linebreak
\includegraphics[width=0.4\textwidth]{os_07.png}
\includegraphics[width=0.4\textwidth,height=7cm]{os_08.png}\\
The wire frame mode can also be toggled globally for all meshes (\textit{Edit->Show Wire frame} or \textit{W}) but that would also 
show the vertices and faces of the reference model.\\ Wait you don't know what vertices and faces are?\\
Let me explain: To represent an object in a virtual 3D environment you need a mesh. A mesh is a collection of multiple vertices 
that form faces together. Vertices are points with coordinates in space. Faces are triangles of connected vertices. Remember 
hating math in school? Well, this is math :(\\
Clipping happens when faces of the reference model intersect with the faces of the outfit.\\ But you may ask yourself:
"\textit{If this is all math, why has no one created a computer program that converts outfits by increasing the mesh area on faces 
that intersect with the reference model by using basic algebra and a good search algorithm?}"\\ The answer is: "\textit{No one has 
the time to do stuff like this. Also only a handful of people using OS are knowledgeable enough in Python or C to do this.}"\\
Enough theory and back to practice! In the wire frame mode you want to search for faces that intersect with the reference model. 
Do note that this \textbf{CAN} help in 80\% of all conversions if done \textbf{CORRECTLY}. Your task will be to check these meshes 
in wire frame mode (one at a time): \textit{boots}, \textit{stockings}, \textit{short} and \textit{belt}. Continue reading after finishing the task.\\
All of these meshes do not have clipping or intersecting faces. Let's look at the last one: \textit{top} together.
The place where you will find faces intersecting is in the area around the neck and on the shoulders.\\
\includegraphics[width=0.3\textwidth]{os_09_textures.png}
\includegraphics[width=0.3\textwidth]{os_09_notextures.png}
\includegraphics[width=0.3\textwidth]{os_09_wireframe.png}\\
As you can hopefully see from the pictures: This sort of clipping is only visible in wire frame mode. Working in wire frame mode is 
only important for the zero-slidered shape as this is the base for everything else. If we fix this now, we minimize possible 
clipping due to sliders or physics. After making sure that the top is also completed make all meshes visible and proceed to the next step.\\
\textbf{Sixth} Conform All again. We conformed the shapes at the beginning when loading the CBBE to Fusion Girl reference model so that 
the sliders will affect the outfit. We want to conform all again because we have a different reference model and want the sliders of 
the Fusion Girl body to affect the outfit.\\
\textbf{Seventh} check \textbf{all} sliders for clipping. I've seen a lot of conversions where this step gets ignored or they don't know 
this exists. What we have done so far is fixing all clipping for the zero-slider preset. If this conversion is private and you use 
the zero-slider preset in game, than you're kinda done and skip this step. If you want to publish this conversion to the nexus or don't 
use the zero-slider preset than continue.\\
This is the part that \textbf{will take the longest}. Like I said at the start of this guide: You need \textbf{patience} and \textbf{time} 
if you want do create a decent conversion. Lets start by looking at the slider panel:

\begin{wrapfigure}[12]{r}{0.6\textwidth}
    \vspace{-10pt}
    \includegraphics[width=0.6\textwidth]{os_10.png}
\end{wrapfigure}
Each slider has a little pencil icon, a check box, a name and the actual slider. Moving the slider will shape the reference model 
based on shape data, the check box enables/disables the slider and the clicking the pencil lets us edit the outfit for the slider.\\
This means that we have to go through each of the slider by clicking the pencil icon and start fixing the clipping. Do note 
that some sliders are more extreme than others and you can ignore a lot of them through experience. This guide has a list of sliders 
at the end with notes of my experience and information about what you can ignore. For now, continue with the example.\\
\linebreak
\linebreak
\linebreak
\linebreak
\linebreak
Let us look at the first slider: \textit{Belly Pregnant}\\
\includegraphics[width=\textwidth]{os_11}\\
As you can see it's not a problem of the outfit clipping through the body but parts of the outfit clipping through each other. 
Always remember that the original mod was not made for Fusion Girl! To fix this simply select the \textit{belt} mesh and increase 
the mesh volume in that area.\\
\includegraphics[width=\textwidth]{os_11_fixed}\\
This slider is very extreme and will always give you some clipping to fix. Your task will be to check every slider you think will have 
clipping.\\ To reduce the time it takes to convert you don't want to actually check every slider but use the information of the 
extreme slider of a group. Let me explain:\\
The \textit{Belly Pregnant} Slider can be considered the extreme slider of the \textit{Belly} group. You just fixed this slider and 
know that the amount of clipping is very small. The remaining sliders of the \textit{Belly} group are: \textit{Belly Big}, 
\textit{Belly Size} and \textit{Belly Tuck}. All of those morph the belly area but not so extreme as the \textit{Belly Pregnant} 
slider, meaning that if you know how the extreme slider morphs and know how each slider works, you will be able to figure out 
what sliders also have clipping without seeing them.\\ It takes time and experience to get good at this so for now, try and 
imagine what belly-sliders will have clipping and check them afterwards. Doing this for every conversion will train your 
\textit{mental picture} of the outfit using different sliders and helps reduce the time it takes for each outfit by \textbf{a lot}.\\
The belly sliders that have clipping are \textit{Belly Size} (on the belt) and \textit{Belly Tuck} (in the crotch area of the shorts).
Do not worry if you didn't got it right the first time as those two have a small amount of clipping.\\
Next up is the \textit{Boobs} group. The extreme slider is \textit{Boobs Yuge} so try the same again.\\
The slider with clipping is: \textit{Boobs Yuge}. But we have another problem! The \textit{Boobs Together} slider doesn't 
introduce clipping but something different.

\begin{wrapfigure}{l}{0.6\textwidth}
    \vspace{-10pt}
    \includegraphics[width=0.6\textwidth]{os_12.png}
\end{wrapfigure}
The sliders have affected the mesh in a way that the mesh starts looking weird. This is not intended and needs to be fixed.
The smooth tool (\textit{shortcut: 5}) can help us here. Active wire frame mode for the mesh, select the smooth tool and begin 
smoothing the area. While using the smooth tool you will notice that the mesh starts to change and the vertices start to align.
The smooth tool can be very powerful but also very useless at the same time. Sometimes the mesh is so deformed that even the 
smooth tool won't help.\\
\linebreak
Your result should look something like this:\\
\includegraphics[width=\textwidth,height=11cm]{os_12_fixed.png}\\
Next up is the \textit{Nipples} group, one that I pretty much always ignore. The slider you will have to look out for is the 
\textit{Nipples Length} slider as this is the extreme slider of the group.\\
The following 11 sliders from \textit{Chest Depth} to \textit{Hips Upper Width} can be quickly checked by starting with 
\textit{Chest Width} and \textit{Waist Chubby}. Those two kinda have the most impact of the 11 sliders and can be used for guidance 
when thinking about how the other 9 sliders affect the outfit. None of them have clipping.\\
We now move from the top part of the body to the lower areas and start with the \textit{Bum} group.\\
This group can get very extreme and you should check \textbf{all of them}.\\ You may have noticed that the upper area has three meshes 
overlaying each other: \textit{shorts} at the bottom, \textit{top} above that and \textit{belt} on top. Three layers of clothing 
will make it hard for the body to clip through but can make the layers intersect each other as well. This happens on the 
\textit{Bum Chubby} slider where the \textit{top} clips through the \textit{belt}.\\
The other bum sliders are fine individually but not when combined. When dealing with this group I always set the \textit{Bum Apple}
slider to 100 and go through each bum slider again.
\includegraphics[width=\textwidth]{os_13.png}\\
In combination with \textit{Bum Apple}, the sliders with clipping are: \textit{Bum Round}, \textit{Bum Chubby}, 
\textit{Bum Crack} and \textit{Bum Size}. You should now also know how to handle the deformation that are visible on the 
\textit{Bum Crack} slider (tip: smooth tool).\\
We continue to go lower and arrive at the two \textit{Calf} Sliders which can often be ignored (also in this case).\\
\textit{Legs Chubby} and \textit{Thighs Size} on the other hand are very important sliders. Like before you should do each one 
individually and than set \textit{Legs Chubby} to 100 and work on \textit{Thighs Size} again. Both combined will often result in 
clipping. Both sliders, even combined, do not clip in this outfit.\\
All sliders till \textit{AbDefintion} can be ignored. Most of them don't concern us or don't even do something (like the arms sliders).
The last groups of sliders are the \textit{Body} and \textit{Abs} groups. The first one changes the whole body while the other one 
changes the abs.\\
You should check \textit{AbDefinition}, \textit{Body Toning}, \textit{XunAbs}, \textit{FFB Fitness 2}, \textit{Athletic} and 
\textit{SeveNBase Bombshell}. The last one will change the body completely. You may ask why I don't include \textit{TigerSanBB}.
The reason is that I don't like it\dots I've never done that slider and no one has ever complained so I guess no one uses that 
slider. It takes too much time and will often destroy the outfit's mesh, idea and spirit. If you wanna do it, 
than \href{https://www.youtube.com/watch?v=VqQJ2wk6ge4&t=21s}{dew it} or \href{https://www.youtube.com/watch?v=dmQDVHjTveI}{dew it}.\\
That's it. You've made it. All sliders should be fixed and you are ready for the last steps.\\
\textbf{Eighth} zap sliders. Zap sliders are wonderful. \href{https://fallout4commands.com/commands/zap-5786997}{\textit{Zap}}
is an in game command that deletes the targeted object. A Zap slider does pretty much the same but instead of deleting a whole object,
it removes vertices from being build in BodySlide. You zap parts of the reference body that are not visible so that they won't clip 
through the outfit when physics starts doing it's thing.\\
To create a zap slider we activate the wire frame mode for all meshes of the outfit and turn on the no textures view. Select the 
reference model ({\color{green}{BaseFBody}}) and select the mask tool (\textit{shortcut: 1}).\\
There are \textbf{three} ways of creating a zap slider. You \textbf{either} mask all parts of the body that you don't want to zap 
\textbf{or} mask all parts of the body that should get zapped and than invert the mask \textbf{or} mask all parts of the body that should get zapped
and create an inverted zap slider. \href{https://idioms.thefreedictionary.com/all+roads+lead+to+Rome}{All roads lead to Rome} 
so try all methods and see what works best for you. I personally prefer the second method.\\
\includegraphics[width=0.5\textwidth,height=5cm]{os_16_front}
\includegraphics[width=0.5\textwidth,height=5cm]{os_16_back}\\
The most important areas you want to zap are the legs+feet and the bum. You will see that I did not masked the thighs because 
the stockings show the body. If you have textures with transparency than you can't zap that area. Another tip is to not go too ham 
on the boobs and the upper back area.\\
Like I said before: I prefer the second method meaning that I have masked the part of the body that should get zapped. I now need 
to invert the mask using either \textit{Ctrl+I} or \textit{Tool->Invert Mask}. To create a zap slider head over to the \textit{Slider} 
menu and select \textit{New Zap Slider}. Be sure to give it a good name like \textit{ZapBody}.
Clear the mask using \textit{Ctrl+A} or \textit{Tool->Clear Mask} and click the pencil icon of the zap slider in the slider list.

\begin{wrapfigure}[11]{l}{0.6\textwidth}
    \vspace{8pt}
    \includegraphics[width=0.6\textwidth]{os_17.png}
\end{wrapfigure}
You open this menu by pressing \textit{Tab} or \textit{Slider->Properties}. This is the properties menu for the zap slider.
If you went for method 3 and want to create an inverted zap slider tick the box for \textit{Invert}.\\
Independent of the method you choose, you will need to make sure the user doesn't mess with the slider so tick \textit{Zapped}
and \textit{Hidden}.\\
\includegraphics[width=0.4\textwidth,height=8cm]{os_17_fixed.png}\\
\textbf{Ninth} bone weighting. You may remember deleting all bones from the project at the start in the first step, we are now going 
to copy the bone weights from the reference model to the current outfit. Proceed to the \textit{Meshes} tab, select all meshes except
the reference model, right click and \textit{Copy Bone Weights}.\\
\includegraphics[width=\textwidth]{os_14-cut.png}\\
This little window will pop up and explain to you what the function \textit{Copy Bone Weights} does. If you have no idea what all 
this text means, don't worry. The important information from that text is that the default values are often sufficient 
and manual tweaking may be required. Just click \textbf{OK} as the default values are good for this outfit.\\
\textbf{Tenth} save the project as a new one. \textbf{DO NOT overwrite to original CBBE BodySlide files!} What you want to do is 
either pressing \textit{Ctrl+Shift+S} or \textit{File->Save Project As}.\\
\includegraphics[width=0.6\textwidth,height=6cm]{obi-wan-meme02.jpg}\\
\includegraphics[width=0.5\textwidth]{os_15-cut.png}
\includegraphics[width=0.5\textwidth]{os_15_fixed-cut.png}\\
This little window will open letting you change different properties before saving.\\
The \textit{Display Name} should be changed to reflect the conversion to the Fusion Girl body. I like to write the name of the 
outfit followed by a "- Fusion Girl" so that it will be "Simply Clothes - Fusion Girl".\\ After setting the Display Name, click the 
\textit{To Project} button and you will see that all text fields in the project field group are now the same as the display name.\\
The original mod has only one outfit so you can now click \textit{Save}. The other different properties and their uses are explained 
in another section of this guide that you can read \textbf{after} finishing this project.\\
\textbf{Eleventh\#1} previewing the outfit. You can now exit Outfit Studio and BodySlide. You have to reopen BodySlide because new projects 
only get loaded on the start of BodySlide. Inside BodySlide select the created project from the list at the top, the name of the project is 
the display name you assigned the last step.\\
Click the preview button and look if the outfit has any clipping with your selected preset (use a preset for the Fusion Girl body not for CBBE).
Also check if you see missing parts of the body. This is the result of bad masking for the zap slider and will be fixed in step twelve.\\
\textbf{Eleventh\#2:} Creating a SliderGroup file. A SliderGroup file can either be created manually or with BodySlide. I recommend 
BodySlide so you don't make any mistakes manually. In BS open the \textit{Group Manager} menu from the button in the top right corner 
and click \textit{Save As}. Give the file a good name, I never have spaces in this file name and will call this one 
\textit{SimplyClothes-FusionGirl}. Doing so will create a new empty SliderGroup file that we can open by clicking the \textit{Browse} 
button and selecting the new file. You \textbf{should always} add the \textit{Fusion Girl} group by typing the name of the Group in the 
textbox left of the \textit{Add Group} button and than clicking that button.\\
\includegraphics[width=\textwidth]{os_18.png}\\
You can now select the group from the \textit{Groups} list and will notice that the elements of the panel in the right are now 
clickable. The \textit{Outfits} panel has a list of all projects loaded in BS. Simply select the Conversion you just created and click 
\textit{<< Add}. Click \textit{Save} and you successfully have added the conversion to the Fusion Girl group. You can now exit the Group Manager.
BS will not have the new group loaded until the next restart but you can reload all groups manually by clicking the lens left of the 
\textit{Group Filer} textfield in the top right corner. Click \textit{Refresh Groups} and select the Fusion Girl group next, there you 
will see that the conversion was added.\\
\textbf{Twelfth} fixing the outfit (\textbf{OPTIONAL}, \textit{skip if you don't have clipping or bad masking}).\\
To re-open the project simply click the second icon right of the \textit{Outfit/Body} Drop down menu or load the project in OS 
using \textit{Ctrl+O} or \textit{File->Load Project}.\\
Fixing bad zap masking:\\
Select the reference model and than the zap slider by clicking the pencil tool and use the 
\textit{Slider->Mask Affected Vertices} function. Next step is to delete the old zap slider using \textit{Slider->Delete Slider}.
You still have the mask so \textbf{either} invert the mask and fix the bad masking \textbf{or} fix the bad masking and than invert.
Either way after fixing recreate the zap slider with the \textit{Zapped} and \textit{Hidden} properties checked.\\
Fixing clipping:\\
This can be a bit time consuming if you don't know where to look. You can load the preset into OS \textit{Slider->Load Preset}.
You can only edit the shape if either all sliders are set to 0 or one slider is selected using the pencil. You will need figure out 
what slider is causing the clipping. Just look at the area of the clipping and find the corresponding slider group. Move the extreme 
slider from 0 to 100 and check if this changes something. Once you narrowed it down to one or two sliders, simply click the pencil and 
edit the mesh. You do not have to copy the bone weights again so just save the project \textit{Ctrl+S}/\textit{File->Save Project}.\\
\textbf{Thirteenth} testing the outfit in game. Before you go and jump in game make sure you build the correct outfit in BodySlide.
If that is done go ahead and start FO4, load a testing save game (or create a new one) and add the outfit to your inventory.\\
Use \textit{help Simply 0 ARMO} to get a list of all items with the signature \textit{ARMO} that contain the name \textit{Simply} and 
use \textit{player.additem} to add the outfit.\\
I advise to start going through the animations in order: standing, walking, running, jumping. See if there is any clipping on the bum 
or the knees. Those are the areas you need to check in game. You should not find anything if following the steps correctly.\\
\textbf{Fourteenth} taking screenshots. You can do this while testing the outfit in game. I'm no screenarcher and the evolution of 
my screenshots are reflected in the mod images of mine. \href{https://steamcommunity.com/sharedfiles/filedetails/?id=172301317}{Here} 
is a very useful guide you might wanna check if you intend to do 'good' screenshots. You can also ask someone nice to help you like 
the nice people in our \href{https://discord.gg/JakcQPN}{discord}. I can't really write anything else here\dots just do some screenshots.
\textbf{Fifteenth} finishing up. You should be done now. You replaced the CBBE body with the FG one, you fixed all clipping for zero-
slidered and different slidered, you tested the outfit, you adjusted stuff and you took screenshots. What's left is depending on your needs.
I did a Simply Clothes Fusion Girl BodySlide Conversion as my first conversion and published it to the nexus. This \textit{little} 
tutorial on converting this specific outfit is now done. You can be proud if everything worked and you have no problems.\\
What's next?\\
Convert more outfits. This time choose one that wasn't converted already and \textbf{keep the difficulty low}. This guide is far from 
finished. You can continue reading but if you wanna stop here: I hope you learned something new today.
\href{https://www.youtube.com/watch?v=3NuFVQk_CCs}{Congratulations}\\