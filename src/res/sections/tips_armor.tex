You have two kinds of armor types on the Nexus: \href{https://www.nexusmods.com/fallout4/mods/10692}{Skimpy with protection} and 
\href{https://www.nexusmods.com/fallout4/mods/13572}{lore-friendly full cover armor}. Let's start with the full cover armor:\\
This kind of armor can come in two variations: \textbf{Either} a complete armor set in slot 33 like the \textit{HN66s SIRIUS.12 Assault Suit} linked 
above \textbf{or} modular with multiple armor pieces in different slots like \href{https://www.nexusmods.com/fallout4/mods/27280}{\textit{HN66s SIRIUS.16 Assault Armor}}.\\
The beauty in this is that you can cheese a lot again, similar to what I said in the Bodysuit section. The full cover complete armor 
set is the easier on and you will have less problems with the body clipping through the armor but each armor piece clipping through 
each other.
\begin{wrapfigure}[30]{l}{0.4\textwidth}
    \vspace{-12pt}
    \includegraphics[width=0.4\textwidth,height=15cm]{tips_armor_01.png}
\end{wrapfigure}
Just take a look at the amount of meshes in the HN66s SIRIUS.12 Assault Suit.\\
Those are \textbf{34} meshes btw. Most of them are not so important like logos or stickers on the suit but in this case you 
have a Bodysuit covering the whole body and than for each body part armor pieces that are all connected.\\
Sure you can easily zap the whole body but you still have to worry about each armor piece.\\
The good thing about all of this is the fact that you have to do the most work on the base shape. You don't have that much to fix 
on different sliders as you zap the body anyway and all that armor gets morphed nicely based on the zero slidered base shape.\\
The skimpy but protective armor is often modular like the Fortaleza Armor Set I linked above. The Fortaleza Armor for example had 
some bodysuits in slot 33 and the rest were all individual armor pieces for L-Arm, R-Arm, L-Leg, R-Leg and Torso. You can't 
copy the reference model for those individual armor pieces meaning that you also can not zap the body. Good for me was the fact 
that each piece had a huge amount of vertices and a very thick structure that went into the skin to reduce clipping.\\
What you should always check when dealing with individual armor pieces is combining them and seeing how the work together. If 
you have one bodysuit and parts for each slot (L-Arm, R-Arm, L-Leg, R-Leg and Torso) in individual projects, you can load the 
bodysuit and add the pieces using \textit{Ctrl+Shift+O} to the project. Only do this \textbf{after} you saved the bodysuit and 
\textbf{do not} save the project with all the pieces in it. This is intended to check how they work together.\\
\includegraphics[width=\textwidth]{armor_header.png}