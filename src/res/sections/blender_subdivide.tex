Sometimes the outfit you are converting has a very low amount of vertices. Examples are 
\href{https://www.nexusmods.com/fallout4/mods/22431}{Elis Armour Compendium} or any vanilla outfit.\\
Blender, and other modeling software, has a feature called \href{https://docs.blender.org/manual/en/latest/modeling/meshes/editing/subdividing/subdivide.html}{Subdivide}
which splits selected edges and faces by cutting them in half, adding new vertices and 
subdividing accordingly the faces involved. It adds resolution to the mesh by dividing faces or edges into 
smaller units.\\
\includegraphics[width=0.5\textwidth]{blender_subdivide_01.png}
\includegraphics[width=0.5\textwidth]{blender_subdivide_02.png}\\
You should only subdivide once and only when the outfit is really blocky like \href{https://speed-new.com/wp-content/uploads/2015/10/34254.jpg}{this}.\\
I will use Blender 2.8 for this guide and work with the \textit{.obj} file instead of the \textit{.nif} file as 
the \href{https://github.com/niftools/blender_nif_plugin}{Blender NIF Plugin} is not yet updated for 2.8.\\
We need to start by exporting the \textit{.obf} file from Outfit Studio so load your project, select the mesh 
you want to export and \textit{Right Click->Export->To Obj}. Keep Outfit Studio and the project open as we will return 
later. Next start Blender:\\
\includegraphics[width=\textwidth]{blender_subdivide_03.png}\\
\linebreak
\linebreak
Blender 2.8 should look something like this. In the top right corner, select the \textit{Cube} mesh and \textit{Right Click->Delete}.
Import the \textit{.obj} file by going to \textit{File->Import->Wavefront (.obj)}.\\
After the mesh loaded in Blender, select the mesh in the top right corner and change from \textit{Object mode} to \textit{Edit mode}
using either \textit{Tab} or the menu in the top left corner of the viewport. In Edit mode go to the menu \textit{Edge} and select 
\textit{Subdivide}.\\
\includegraphics[width=0.6\textwidth]{blender_subdivide_06}
\includegraphics[width=0.4\textwidth]{blender_subdivide_05}\\
Now you can export the mesh again using \textit{File->Export->Wavefront (.obj)}. It is important that you change the export settings
in the botton right corner. You will have to tick \textit{Selection Only} so that only the selected mesh gets exported.\\
Return to Outfit Studio after exporting and go to \textit{File->Import->From OBJ}. Outfit Studio will ask you to provide a name for the 
mesh and make sure to give it a unique name. You will also notice that the imported mesh does not have a texture. Select the original 
mesh and open the Properties menu and copy the path to the Material. Paste this path in the exported mesh's Material propertie and click ok.\\
You should be done now and have a new mesh with a higher vertices count. Remember do delete the old mesh as you won't need it anymore.\\
A lot of things can go wrong in this process and it may not even work for your mesh at all. You could also try getting Blender 2.6 and the 
Blender NIF Plugin and try exporting and importing the nif file instead of the obj file to see if that works better.