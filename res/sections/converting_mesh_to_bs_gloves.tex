Creating BodySlide files for gloves is on the easier side of things but still often ignored. Most CBBE BodySlide projects do not 
include the gloves but they are important to convert due to the different skeleton and bones CBBE and FG uses.\\
\textbf{First} load the outfit. To load the outfit go to \textit{File->Load Outfit} and navigate to the nif file. It is important 
to select the nif file inside the Fallout 4 data folder and not in the MO2 folder. You should also change the Display Name to 
something meaningful like \textit{NameOfGloves - Fusion Girl}.\\
\textbf{Second} delete old bones. Head over to the \textit{Bones} tab and delete all bones from Project \textit{Right Click->Delete->From Project}.\\
\textbf{Third} load the hands reference model. The hands model is separate from the body so go to \textit{File->Import->From Nif} and
load \textit{data/meshes/actors/character/characterassets/FemaleHands.nif}.\\
\textbf{Fourth} fix clipping. Select the gloves mesh and fix clipping (I advise to disable x-mirror \textit{Edit->X Mirror}).\\
\textbf{Fifth} copy bone weights. Select the \textit{NewFHands} mesh and \textit{Right Click->Set Reference}. The name of the mesh
should now be in {\color{green}{green}} and you can select the gloves mesh and \textit{Right Click->Copy Bone Weights}.\\
\textbf{Sixth} (optional) if the gloves go beyond the hands and also cover the forearm than you need to follow this step, if they just 
cover the hands than proceed to step seven.\\ You have to select the \textit{NewFHands} mesh and \textit{Right Click->Set Reference}. This 
will make the text of the mesh white. You need to do this because we now load the Fusion Girl reference body which would override other 
reference models as only one reference mesh can exist. So just go to \textit{File->Load Reference} and select \textit{Fusion Girl}. Select 
the gloves again and \textit{Right Click->Copy Bone Weights}. Delete the Fusion Girl body \textit{BaseFBody, Right-Click->Delete} and 
make \textit{NewFHands} the reference model again using \textit{Right Click->Set Reference}.\\
\textbf{Seventh} save the project. Using \textit{Ctrl+Shift+S} save the project and \textbf{copy the reference shape into output}.