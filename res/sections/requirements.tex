With that introduction out of the way we can get down to business. In this section we will download all necessary 
programs, mods and assets you may need. All of them \textit{(except the game)} are free and most 
are also open source. \textbf{BEFORE} you go and download everything in the list below, read the instructions first.
\begin{center}
\vspace{0.5cm}
\begin{tabular}{| p{0.75\textwidth} | c |}
    Name & Download \\ [1ex]
    \hline
    Fallout 4 (all DLCs) & \href{https://store.steampowered.com/app/377160/Fallout_4/}{Steam}  \\ [1ex]
    Mod Organizer 2 & \href{https://www.nexusmods.com/skyrimspecialedition/mods/6194}{Nexus}/\href{https://github.com/ModOrganizer2/modorganizer/releases}{Github} \\ [1ex]
    BodySlide and Outfit Studio & \href{https://www.nexusmods.com/fallout4/mods/25}{Nexus} \\ [1ex]
    FO4Edit & \href{https://www.nexusmods.com/fallout4/mods/2737}{Nexus}/\href{https://github.com/TES5Edit/TES5Edit/releases}{Github} \\ [1ex]
    zEdit & \href{https://github.com/z-edit/zedit/releases}{Github} \\ [1ex]
    NifSkope (optional) & \href{http://niftools.sourceforge.net/wiki/NifSkope}{Sourceforge}/\href{https://github.com/niftools/nifskope/releases}{Github} \\ [1ex]
    Blender 2.8 (optional) & \href{https://www.blender.org/download/}{Blender}/\href{https://store.steampowered.com/app/365670/Blender/}{Steam} \\ [1ex]
\end{tabular}
\vspace{0.5cm}
\end{center}

\begin{flushleft}
    \href{https://www.youtube.com/watch?v=YPN0qhSyWy8}{\textbf{Fallout 4}} is a no brainer but \textbf{you need all DLCs} for modding. A huge amount of
    mods on the Nexus do not support a bare bone Fallout 4.\\
    \vspace{0.1cm}
    \textbf{Mod Organizer 2} is the best organizer out there. The best feature MO2 has is the Virtual File System. The mods you install in MO2 are not in the game folder but in a special 
    MO2 mods folder that gets virtually loaded for the game once you start it. Meaning that you have a non destructive virtual environment to work with.
    Even if your MO2 becomes a total mess, your game folder stays clean.\\
    \vspace{0.1cm}
    \textbf{BodySlide and Outfit Studio} are bundled together and will be the main tools for this guide. Originally made for CBBE, BodySlide will refit 
    every armor to your selected body shape while Outfit Studio helps you create armor for Bethesda games.\\
    \vspace{0.1cm}
    \textbf{FO4Edit} is the FO4 release of the xEdit series and will be used to manipulate esps and view information about the outfits we convert.\\
    \vspace{0.1cm}
    \textbf{zEdit} is a powerful alternative to xEdit and features very small load times making it a good tool for just viewing information \textit{I also helped a bit during development :)}\\
    \vspace{0.1cm}
    \textbf{NifSkope} \textit{(optional)} is a tool that lets us view \textit{.nif} files. Nif files contain the mesh and bones of a 3D model.\\
    \vspace{0.1cm}
    \textbf{Blender 2.8} \textit{(optional)} is the best free and open source 3D modeling tool out there. The 2.8 release just went live and we can import and export nif files using the \href{https://github.com/niftools/blender_nif_plugin}{blender nif plugin} by the NifTools Project.
\end{flushleft}