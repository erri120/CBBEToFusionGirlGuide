\href{https://www.creationkit.com/fallout4/index.php?title=Biped_Slots}{Biped Slots} determine what body parts the outfit covers. 
This element is part of an \href{https://www.creationkit.com/fallout4/index.php?title=ArmorAddon}{ARMA} record which is referenced by 
an \href{https://www.creationkit.com/fallout4/index.php?title=Armor}{ARMO} record. Sounds confusing? Let me explain:\\
Adding an outfit into FO4 requires an esp. Using the creationkit or xEdit/zEdit you can create an esp and add an ARMO record to it.
This ARMO record contains all information about an outfit. ARMO stands for Armor and ARMA stands for ArmorAddon. The ARMO record also 
contains the Editor ID of the ARMA record. You have one ARMA record for every ARMO record, no more, no less. The ARMA record contains 
information about what nif file to use and what Biped Slot the armor/outfit uses.\\
"This is all nice and good, but what has this to do with converting outfits?". you may ask. Well it is important to understand 
what type of Biped Slot the outfit you are converting has. The most important, is slot \textbf{33 - BODY}.\\
Remember when saving a project that you have the option to copy the reference model into the output file? If the outfit is in slot 
33 than you \textbf{must} copy the reference model. If you don't have a BODY in the slot for the BODY than you won't see the body.
Simple, right? You can try and not copy the ref model for a slot 33 outfit and see what happens.\\
This also means that you \textbf{should not} copy the ref model for everything else. For example: If you have an outfit in slot 
41 - [A] Torso than you don't want to copy the ref model because doing so will make the output file 3x larger and can introduce
clipping.\\
On the topic of clipping: \textbf{You can't zap a non existing ref model. Only use zap sliders in a project if the outfit is in 
slot 33.}\\
Lets have a look inside an esp using zEdit (using \textit{SimplyClothes} as an example):\\
\includegraphics[width=\textwidth]{slots_01.png}\\
You can see that you have one ARMO record and five ARMA records.\\
\linebreak
\includegraphics[width=\textwidth]{slots_02.png}\\
Inside the ARMO record we can find out what ARMA record it uses.\\
\linebreak
\includegraphics[width=\textwidth]{slots_03.png}\\
The ARMA record shows that the outfit uses slot 33.\\
\linebreak
\includegraphics[width=\textwidth]{slots_04.png}\\
You can also get the slots in the ARMO record but I prefer to look at the ARMA record so that I also know what nif file it uses.
(\textit{Note that in this case the SimplySuit shows more slots than the ARMA record because it is made up of multiple parts 
like Boots,Shorts,\dots})